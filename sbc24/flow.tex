\section{Trust Flow}
  We can now define indirect trust from $A$ to $B$.
  \import{common/definitions/}{indirecttrust.tex}
  \noindent Note that $Tr_{A \rightarrow B} \geq DTr_{A \rightarrow B}$. The next result shows $Tr_{A \rightarrow B}$ is
  finite.
  \import{fc17/theorems/}{convergencetheorem.tex}
  \import{common/proofsketches/}{convergenceproofsketch.tex}
  Proofs of all theorems can be found in Appendix A\ifdefined\proceedings \ of the full version \cite{trustisrisk}\fi.

  In the setting of \texttt{TransitiveGame(}$\mathcal{G}$\texttt{,}$A$\texttt{,}$B$\texttt{)} and $j$ being a turn in which
  the game has converged, we use the notation $Loss_A = Loss_{A, j}$. $Loss_A$ is not the same for repeated executions of this
  kind of game, since the order in which players are chosen may differ between executions and conservative players can choose
  which incoming direct trusts they will steal and how much from each.

  Let $G$ be a weighted directed graph. We investigate the maximum flow on it. For an introduction to maximum
  flows see Introduction to Algorithms, p. 708 \cite{clrs}. Considering each edge's capacity as its weight, a flow
  assignment $X = [x_{vw}]_{\mathcal{V} \times \mathcal{V}}$ with source $A$ and sink $B$ is valid when:
  \begin{equation}
  \label{flow1}
    \forall (v, w) \in \mathcal{E}, x_{vw} \leq c_{vw} \mbox{ and}
  \end{equation}
  \begin{equation}
  \label{flow2}
    \forall v \in \mathcal{V} \setminus \{A,B\}, \sum\limits_{w \in N^{+}(v)}x_{wv} = \sum\limits_{w \in N^{-}(v)}x_{vw}
    \enspace.
  \end{equation}
  The flow value is $\sum\limits_{v \in N^{+}\left(A\right)}x_{Av} = \sum\limits_{v \in N^{-}\left(B\right)}x_{vB}$. We do not
  suppose skew symmetry in $X$. There exists an algorithm $MaxFlow\left(A, B\right)$ that returns the maximum possible flow
  from $A$ to $B$.  This algorithm needs full knowledge of the graph and runs in $O\left(|\mathcal{V}||\mathcal{E}|\right)$
  time \cite{maxflownm}. We refer to the flow value of $MaxFlow\left(A, B\right)$ as $maxFlow\left(A, B\right)$.

  We will now introduce two lemmas that will be used to prove one of the central results of this work, the Trust Flow
  theorem.
  \import{fc17/lemmas/}{flowgamelemma.tex}
  \import{common/proofsketches/}{flowgameproofsketch.tex}
  \import{fc17/lemmas/}{gameflowlemma.tex}
  \import{common/proofsketches/}{gameflowproofsketch.tex}
  \import{common/theorems/}{trustflowtheorem.tex}
  \import{common/proofs/}{trustflowproof.tex}

  \noindent Note that the maxFlow is the same in the following two cases: If a player chooses the evil strategy and if that
  player chooses a variation of the evil strategy where she does not nullify her outgoing direct trust.

  Further justification of trust transitivity through the use of $MaxFlow$ can be found in the sociological work by Karlan et
  al. \cite{kmrs} where a direct correspondence of maximum flows and empirical trust is experimentally validated.

  Here we see another important theorem that gives the basis for risk-invariant transactions between different, possibly
  unknown, parties.
  \import{common/theorems/}{riskinvtheorem.tex}
  \import{common/proofs/}{riskinvproof.tex}
  \noindent $A$ can reduce her outgoing direct trust in a manner that achieves (\ref{primetrust}), since
  $maxFlow\left(A, B\right)$ is continuous with respect to $A$'s outgoing direct trusts.
