\section{Utility Function Properties}
  Let $v \in \mathcal{P}$. The only property that any utility function should satisfy is that having more desires satisfied
  and less desires unsatisfied leads to a higher utility.
  More formally, consider two executions of the game (1) and (2):
  \begin{equation*}
    \begin{rcases}
    \begin{aligned}
      SatDesires_1\left(v\right) &\subseteq SatDesires_2\left(v\right) \\
      UnsatDesires_2\left(v\right) &\subseteq UnsatDesires_1\left(v\right)
    \end{aligned}
    \end{rcases}
    \Rightarrow u_{v,1} \leq u_{v,2} \enspace.
  \end{equation*}
  If one of the two subset is strict, then the inequality becomes strict as well.

  As an example, consider two executions of the game (1) and (2) in which $v$ was input exactly the same desires and satisfied
  the same desires, except for $\left(idx, d\right)$, which was satisfied only in game (2) and remained unsatisfied in game
  (1). Then $u_{v,1} < u_{v,2}$. Going in some detail:
  \begin{gather*}
    \mbox{Let } Desires_1\left(v\right) = Desires_2\left(v\right) = \{\left(idx, d\right)\} \cup Rest \enspace. \\
    \mbox{It is: } \:\:
    \begin{aligned}
      SatDesires_1\left(v\right) &= Rest \enspace, \\
      UnsatDesires_1\left(v\right) &= Rest \cup \{\left(idx, d\right)\} \enspace, \\
      SatDesires_2\left(v\right) &= Rest \cup \{\left(idx, d\right)\} \enspace\mbox{and} \\
      UnsatDesires_2\left(v\right) &= Rest \enspace,
    \end{aligned} \\
    \mbox{thus:   } \:\: 
    \begin{rcases}
    \begin{aligned}
      SatDesires_1\left(v\right) &\subset SatDesires_2\left(v\right) \\
      UnsatDesires_2\left(v\right) &\subset UnsatDesires_1\left(v\right)
    \end{aligned}
    \end{rcases}
    \Rightarrow u_{v,1} < u_{v,2} \enspace.
  \end{gather*}
