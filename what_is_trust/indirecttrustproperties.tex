\section{Desired Properties for Indirect Trust}
  \begin{enumerate}
    \item $Tr_{Alice \rightarrow Bob} \geq DTr_{Alice \rightarrow Bob}$
    \item If universe (1) and (2) are identical except for $DTr_{Alice \rightarrow Bob}$, then
      \begin{equation*}
	Tr^2_{Alice \rightarrow Bob} = Tr^1_{Alice \rightarrow Bob} - DTr^1_{Alice \rightarrow Bob} + DTr^2_{Alice \rightarrow
	Bob} \enspace.
      \end{equation*}
    \item Consider an indexed desire $\left(idx, d\right)$ $Alice$ has. Let $p\left(idx, d, Alice\right)$ be a function that
    returns the player that $Alice$ should rationally delegate the calculation of $s\left(idx, d, Alice\right)$ to.
      \begin{enumerate}
        \item If a player is cheaper and more trustworthy than all other players, delegate the calculation to him.
        \begin{gather*}
	  \exists Bob \in \mathcal{P}: \forall Charlie \in \mathcal{P} \setminus \{Bob\} \\
	  \left(c\left(idx, d, Alice, Bob\right) < c\left(idx, d, Alice, Charlie\right) \wedge \right. \\
	  \left. \wedge \ Tr_{Alice \rightarrow Bob} > Tr_{Alice \rightarrow Charlie}\right) \Rightarrow \\
	  \Rightarrow p\left(idx, d, Alice\right) = Bob \enspace.
	\end{gather*}
	\item If there exists a player $Bob$ that is both cheaper and more trustworthy than $Charlie$, do not delegate the
	calculation to $Charlie$.
	\begin{gather*}
	  \exists Bob, Charlie \in \mathcal{P}: \\
	  \left(c\left(idx, d, Alice, Bob\right) < c\left(idx, d, Alice, Charlie\right) \wedge \right. \\
	  \left. \wedge \ Tr_{Alice \rightarrow Bob} > Tr_{Alice \rightarrow Charlie}\right) \Rightarrow \\
	  \Rightarrow p\left(idx, d, Alice\right) \neq Charlie \enspace.
	\end{gather*}
      \end{enumerate}
      Note that the first property can be deduced from the second.
  \end{enumerate}

  Let $x = \left(idx, d, Alice\right)$. Several ideas exist as to what rules $p\left(x\right)$ should satisfy:
  \begin{enumerate}
    \item Indirect trust towards $Bob$ is required to be greater than the cost of the calculation requested by $Bob$ in order
    for $Alice$ to delegate $s\left(idx, d, Alice\right)$ to him. If there exist multiple players that $Alice$ indirectly
    trusts more than their cost, then the cheapest one is chosen. If multiple trustworthy enough players have the same cost,
    the most trustworthy one is chosen.
    \begin{align*}
      & c\left(x, Charlie\right) > Tr_{Alice \rightarrow Charlie} & \Rightarrow p\left(x\right) \neq Charlie \\
      \mathrlap{\mbox{For the following, } \forall v \in \{Bob, Charlie\} \ c\left(x, v\right) \leq Tr_{Alice \rightarrow v}
      \enspace.} \\
      & c\left(x, Bob\right) < c\left(x, Charlie\right) & \Rightarrow p\left(x\right) \neq Charlie  \\
      &
      \begin{rcases}
        c\left(x, Bob\right) = c\left(x, Charlie\right) \\
        Tr_{Alice \rightarrow Bob} > Tr_{Alice \rightarrow Charlie}
      \end{rcases}
      & \Rightarrow p\left(x\right) \neq Charlie
    \end{align*}
    \item Similarly to the previous idea, the indirect trust must exceed the cost. In this case however, in case of multiple
    trustworthy and cheap players, the indirect trust is considered before the cost.
    \begin{align*}
      & c\left(x, Charlie\right) > Tr_{Alice \rightarrow Charlie} & \Rightarrow p\left(x\right) \neq Charlie \\
      \mathrlap{\mbox{For the following, } \forall v \in \{Bob, Charlie\} \ c\left(x, v\right) \leq Tr_{Alice \rightarrow v}
      \enspace.} \\
      & Tr_{Alice \rightarrow Bob} > Tr_{Alice \rightarrow Charlie} & \Rightarrow p\left(x\right) \neq Charlie \\
      &
      \begin{rcases}
        Tr_{Alice \rightarrow Bob} = Tr_{Alice \rightarrow Charlie} \\
        c\left(x, Bob\right) < c\left(x, Charlie\right)
      \end{rcases}
      & \Rightarrow p\left(x\right) \neq Charlie
    \end{align*}
    \item The player with the lowest difference between indirect trust and cost is chosen.
    \begin{equation*}
      p\left(x\right) = \argmin\limits_{v \in \mathcal{P}}\left(Tr_{Alice \rightarrow v} - c\left(x, v\right)\right)
    \end{equation*}
    Note that this last approach constitutes another direction, which departs from choosing the cheapest and most trustworthy
    vendor, opting for the one whose price and trustworthiness match. It evidently does not follow the property (3).
  \end{enumerate}
