\section{Trust Transitivity}
  In this section we define some strategies for the players and the Transitive Game, the worst-case scenario for an honest
  player when another player plays maliciously.
  \subimport{ecescon17/definitions/}{idlestrategy.tex}
  \subimport{ecescon17/definitions/}{evilstrategy.tex}
  \subimport{ecescon17/definitions/}{conservativestrategy.tex}
  The rationale behind this strategy arises from a real-world common situation. Suppose there are a client, an
  intermediary and a producer. The client entrusts some value to the intermediary so that the latter can buy the desired
  product from the producer and deliver it to the client. The intermediary in turn entrusts an equal value to the
  producer, who needs the value upfront to be able to complete the production process. However the producer eventually
  does not give the product neither reimburses the value, due to bankruptcy or decision to exit the market with an unfair
  benefit. The intermediary can choose either to reimburse the client and suffer the loss, or refuse to return the money
  and lose the client's trust. The latter choice for the intermediary is exactly the conservative strategy. It is used
  throughout this work as a strategy for all the intermediary players because it models effectively the worst-case
  scenario that a client can face after an evil player decides to steal everything she can and the rest of the players do
  not engage in evil activity.


  The Transitive Game is the special game of the Trust is Risk Game in which one player is Idle, one player is Evil and the
  rest are Conservative. An example follows.
  \subimport{ecescon17/figures/}{transitivegameexample.tikz}

  \noindent In turn 0, there is already a network in place. All players apart from $A$ and $B$ follow the conservative
  strategy. Each conservative player can be in one of three states: Happy, Angry or Sad. Happy players have 0 loss, Angry
  players have positive loss and positive incoming direct trust, thus are able to replenish their loss at least in part; Sad
  players have positive loss, but 0 incoming direct trust, thus they cannot replenish the loss.

  Let $j_0$ be the first turn on which $B$ is chosen to play. Until then, all players will pass their turn since nothing has
  been stolen yet \cite{trustisrisk}. Moreover, let $v = Player(j)$. The Transitive Game generates turns:
  \begin{align}
     Turn_j = \bigcup\limits_{w \in N^{-}\left(v\right)_{j-1}}\{Steal\left(y_w,w\right)\} \enspace, & \mbox{ where} \\
     \sum\limits_{w \in N^{-}\left(v\right)_{j-1}}y_w = \min\left(in_{v, j-1}, Dmg_{v, j}\right) \enspace. &
  \end{align}

  \noindent We see that if $Dmg_{v, j} = 0$, then $Turn_j = \emptyset$. From the definition of $Dmg_{v,j}$ and knowing that no
  strategy in this case can increase any direct trust, we see that $Dmg_{v,j} \geq 0$. Also $Loss_{v,j} \geq 0$ because
  if $Loss_{v,j} < 0$, then $v$ has stolen more value than she has been stolen, thus she would not be following the
  conservative strategy.
