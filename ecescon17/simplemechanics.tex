\section{Mechanics}
  We now trace $Alice$'s steps from joining the network to successfully completing a purchase. Suppose initially all her
  coins, say $10\bitcoin$, are under her exclusive control.

  Two trustworthy friends, $Bob$ and $Charlie$, persuade her to try out Trust is Risk. She installs the Trust is Risk wallet
  and migrates the $10\bitcoin$ from her regular wallet, entrusting $2\bitcoin$ to $Bob$ and $5\bitcoin$ to $Charlie$. She now
  exclusively controls $3\bitcoin$. She is risking $7\bitcoin$ to which she has full but not exclusive access in exchange for
  being part of the network.

  A few days later, she discovers a shoes shop owned by $Dean$, also a member of Trust is Risk. She finds a pair of shoes that
  costs $1\bitcoin$ and checks $Dean$'s trustworthiness through her new wallet. Suppose $Dean$ is deemed trustworthy up to
  $5\bitcoin$. Since $1\bitcoin < 5\bitcoin$, she confidently proceeds to purchase the shoes with her new wallet.

  She can then see in her wallet that her exclusive coins have remained $3\bitcoin$, the coins entrusted to $Charlie$ have
  been reduced to $4\bitcoin$ and $Dean$ is entrusted $1\bitcoin$, equal to the value of the shoes. Also, her purchase is
  marked as pending. If she checks her trust towards $Dean$, it still is $5\bitcoin$. Under the hood, her wallet redistributed
  her entrusted coins in a way that ensures $Dean$ is directly entrusted with coins equal to the value of the purchased item
  and that her reported trust towards him has remained invariant.

  Eventually the shoes reach $Alice$. $Dean$ chooses to redeem $Alice$'s entrusted coins, so her wallet does not show any
  coins entrusted to him. Through her wallet, she marks the purchase as successful. This lets the system replenish the reduced
  trust to $Bob$ and $Charlie$, setting the entrusted coins to $2\bitcoin$ and $5\bitcoin$ respectively once again.  $Alice$
  now exclusively owns $2\bitcoin$. Thus, she can now use a total of $9\bitcoin$ as expected, since the shoes cost
  $1\bitcoin$.
