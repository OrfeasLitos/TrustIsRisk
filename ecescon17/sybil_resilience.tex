\section{Sybil Resilience}
  One of our aims is to mitigate Sybil attacks \cite{sybilattack} whilst maintaining decentralized autonomy \cite{dionyziz}.
  We begin by extending the definition of indirect trust.
  \subimport{common/definitions/}{indirecttrustmultiplayer.tex}
  \noindent We now extend the Trust Flow theorem to many players.
  \subimport{common/theorems/}{multiplayertrustflowtheorem.tex}
  \noindent We now define several useful notions to tackle the problem of Sybil attacks. Let Eve be a possible attacker.
  \subimport{ecescon17/definitions/}{corrupted.tex}
  \subimport{ecescon17/definitions/}{sybil.tex}
  \subimport{common/definitions/}{collusion.tex}
  \subimport{common/figures/}{collusion.tikz}
  Players $\mathcal{V} \setminus (\mathcal{B} \cup \mathcal{C})$ perceive the collusion as independent players with a distinct
  strategy each, whereas in reality they are all subject to a single strategy dictated by Eve.
  \subimport{ecescon17/theorems/}{sybilrestheorem.tex}
  We saw that controlling $|\mathcal{C}|$ is irrelevant for Eve, thus Sybil attacks are pointless. The theorem does not guard
  against deception attacks though. A malicious player can make many accounts, use them legitimately to inspire the confidence
  of others and then switch to the evil strategy, defrauding everyone that trusted the fabricated identities. These identities
  correspond to the corrupted set of players and not to the Sybil set because they have direct incoming trust from outside the
  collusion.

  In conclusion, we have delivered on our promise of a Sybil-resilient decentralized financial trust system with invariant
  risk for purchases.

