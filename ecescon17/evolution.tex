\section{Evolution of Trust}
  Trust is Risk is a game that runs indefinitely. In each turn, a player is chosen, decides what to play and, if valid, the
  chosen turn is executed.
  \subimport{ecescon17/definitions/}{turns.tex}
  We use $prev\left(j\right)$ to denote the previous turn played by $Player(j)$. If she has not played before,
  $prev\left(j\right) = 0$.
  \subimport{common/definitions/}{damage.tex}
  \subimport{common/definitions/}{history.tex}
  \noindent Knowledge of the initial graph $\mathcal{G}_0$, all players' initial capital and the history amount to full
  comprehension of the evolution of the game. Building on the example of Fig.~\ref{fig:utxo}, we can see the resulting graph
  when $D$ plays
  \begin{equation}
  \label{turnexample}
     Turn_1 = \{Steal\left(1, A\right), Add\left(4, C\right), Add\left(-1, B\right)\} \enspace.
  \end{equation}
  \subimport{ecescon17/figures/}{turnexample.tikz}

  \noindent We now define the Trust is Risk Game formally. We assume players are chosen so that, after her turn, a player will
  eventually play again later.
  \subimport{ecescon17/algorithms/}{trustisriskgame.tex}

  \noindent \texttt{strategy[}$A$\texttt{]()} provides player $A$ with full knowledge of the game, except for the capitals of
  other players. This assumption may not be always realistic. \texttt{executeTurn()} checks the validity of \texttt{Turn} and
  substitutes it with an empty turn if invalid. Subsequently, it creates the new graph $\mathcal{G}_j$ and updates the
  history accordingly.
